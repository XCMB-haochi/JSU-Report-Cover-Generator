\documentclass[a4paper,12pt]{article}
\usepackage{ctex}
\usepackage{graphicx}
\usepackage{geometry}

% 页面设置
\geometry{left=2.5cm, right=2.5cm, top=2.5cm, bottom=2.5cm}

% 字体设置 - 全部使用黑体
\setCJKfamilyfont{hei}{SimHei}
\newcommand{\hei}{\CJKfamily{hei}}

% 字号定义
\newcommand{\yihao}{\fontsize{26pt}{39pt}\selectfont}
\newcommand{\sanhao}{\fontsize{16pt}{24pt}\selectfont}

% 固定宽度标签
\newcommand{\infolabel}[1]{\makebox[5em][s]{#1}}

\begin{document}
\pagestyle{empty}

\begin{center}

% 标题
\vspace*{1cm}
{\yihao\hei 江\ 苏\ 大\ 学\ 课\ 程\ 报\ 告}

\vspace{2cm}

% 校徽
\includegraphics[width=5cm]{UJS-source/jsulogo绿色.pdf}

\vspace{2cm}

% 题目
{\sanhao\hei SPARTA-Net:基于场景感知生成对抗网络的Radio Map补全方法}

\vspace{1.5cm}

\end{center}

% 信息区域
\begin{flushleft}
\hspace{3cm}
\begin{minipage}{12cm}
{\sanhao\hei
\infolabel{学号}:\underline{\makebox[7.5cm]{3240601042}}\\[0.8cm]
\infolabel{班级}:\underline{\makebox[7.5cm]{通信2402}}\\[0.8cm]
\infolabel{姓名}:\underline{\makebox[7.5cm]{徐奕博}}\\[0.8cm]
\infolabel{报告类型}:\underline{\makebox[7.5cm]{智能通信技术基础期末任务}}
}
\end{minipage}
\end{flushleft}

\vfill

% 底部日期
\begin{center}
{\sanhao\hei 2025年11月}
\end{center}

\vspace{1cm}

\end{document}
